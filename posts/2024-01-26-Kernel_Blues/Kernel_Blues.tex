% Options for packages loaded elsewhere
\PassOptionsToPackage{unicode}{hyperref}
\PassOptionsToPackage{hyphens}{url}
\PassOptionsToPackage{dvipsnames,svgnames,x11names}{xcolor}
%
\documentclass[
  letterpaper,
  DIV=11,
  numbers=noendperiod]{scrartcl}

\usepackage{amsmath,amssymb}
\usepackage{lmodern}
\usepackage{iftex}
\ifPDFTeX
  \usepackage[T1]{fontenc}
  \usepackage[utf8]{inputenc}
  \usepackage{textcomp} % provide euro and other symbols
\else % if luatex or xetex
  \usepackage{unicode-math}
  \defaultfontfeatures{Scale=MatchLowercase}
  \defaultfontfeatures[\rmfamily]{Ligatures=TeX,Scale=1}
\fi
% Use upquote if available, for straight quotes in verbatim environments
\IfFileExists{upquote.sty}{\usepackage{upquote}}{}
\IfFileExists{microtype.sty}{% use microtype if available
  \usepackage[]{microtype}
  \UseMicrotypeSet[protrusion]{basicmath} % disable protrusion for tt fonts
}{}
\makeatletter
\@ifundefined{KOMAClassName}{% if non-KOMA class
  \IfFileExists{parskip.sty}{%
    \usepackage{parskip}
  }{% else
    \setlength{\parindent}{0pt}
    \setlength{\parskip}{6pt plus 2pt minus 1pt}}
}{% if KOMA class
  \KOMAoptions{parskip=half}}
\makeatother
\usepackage{xcolor}
\setlength{\emergencystretch}{3em} % prevent overfull lines
\setcounter{secnumdepth}{-\maxdimen} % remove section numbering
% Make \paragraph and \subparagraph free-standing
\ifx\paragraph\undefined\else
  \let\oldparagraph\paragraph
  \renewcommand{\paragraph}[1]{\oldparagraph{#1}\mbox{}}
\fi
\ifx\subparagraph\undefined\else
  \let\oldsubparagraph\subparagraph
  \renewcommand{\subparagraph}[1]{\oldsubparagraph{#1}\mbox{}}
\fi


\providecommand{\tightlist}{%
  \setlength{\itemsep}{0pt}\setlength{\parskip}{0pt}}\usepackage{longtable,booktabs,array}
\usepackage{calc} % for calculating minipage widths
% Correct order of tables after \paragraph or \subparagraph
\usepackage{etoolbox}
\makeatletter
\patchcmd\longtable{\par}{\if@noskipsec\mbox{}\fi\par}{}{}
\makeatother
% Allow footnotes in longtable head/foot
\IfFileExists{footnotehyper.sty}{\usepackage{footnotehyper}}{\usepackage{footnote}}
\makesavenoteenv{longtable}
\usepackage{graphicx}
\makeatletter
\def\maxwidth{\ifdim\Gin@nat@width>\linewidth\linewidth\else\Gin@nat@width\fi}
\def\maxheight{\ifdim\Gin@nat@height>\textheight\textheight\else\Gin@nat@height\fi}
\makeatother
% Scale images if necessary, so that they will not overflow the page
% margins by default, and it is still possible to overwrite the defaults
% using explicit options in \includegraphics[width, height, ...]{}
\setkeys{Gin}{width=\maxwidth,height=\maxheight,keepaspectratio}
% Set default figure placement to htbp
\makeatletter
\def\fps@figure{htbp}
\makeatother

\KOMAoption{captions}{tableheading}
\makeatletter
\makeatother
\makeatletter
\makeatother
\makeatletter
\@ifpackageloaded{caption}{}{\usepackage{caption}}
\AtBeginDocument{%
\ifdefined\contentsname
  \renewcommand*\contentsname{Table of contents}
\else
  \newcommand\contentsname{Table of contents}
\fi
\ifdefined\listfigurename
  \renewcommand*\listfigurename{List of Figures}
\else
  \newcommand\listfigurename{List of Figures}
\fi
\ifdefined\listtablename
  \renewcommand*\listtablename{List of Tables}
\else
  \newcommand\listtablename{List of Tables}
\fi
\ifdefined\figurename
  \renewcommand*\figurename{Figure}
\else
  \newcommand\figurename{Figure}
\fi
\ifdefined\tablename
  \renewcommand*\tablename{Table}
\else
  \newcommand\tablename{Table}
\fi
}
\@ifpackageloaded{float}{}{\usepackage{float}}
\floatstyle{ruled}
\@ifundefined{c@chapter}{\newfloat{codelisting}{h}{lop}}{\newfloat{codelisting}{h}{lop}[chapter]}
\floatname{codelisting}{Listing}
\newcommand*\listoflistings{\listof{codelisting}{List of Listings}}
\makeatother
\makeatletter
\@ifpackageloaded{caption}{}{\usepackage{caption}}
\@ifpackageloaded{subcaption}{}{\usepackage{subcaption}}
\makeatother
\makeatletter
\@ifpackageloaded{tcolorbox}{}{\usepackage[many]{tcolorbox}}
\makeatother
\makeatletter
\@ifundefined{shadecolor}{\definecolor{shadecolor}{rgb}{.97, .97, .97}}
\makeatother
\makeatletter
\makeatother
\ifLuaTeX
  \usepackage{selnolig}  % disable illegal ligatures
\fi
\IfFileExists{bookmark.sty}{\usepackage{bookmark}}{\usepackage{hyperref}}
\IfFileExists{xurl.sty}{\usepackage{xurl}}{} % add URL line breaks if available
\urlstyle{same} % disable monospaced font for URLs
\hypersetup{
  pdftitle={Kernel Blues},
  colorlinks=true,
  linkcolor={blue},
  filecolor={Maroon},
  citecolor={Blue},
  urlcolor={Blue},
  pdfcreator={LaTeX via pandoc}}

\title{Kernel Blues}
\author{}
\date{2024-01-26}

\begin{document}
\maketitle
\ifdefined\Shaded\renewenvironment{Shaded}{\begin{tcolorbox}[frame hidden, sharp corners, borderline west={3pt}{0pt}{shadecolor}, breakable, enhanced, boxrule=0pt, interior hidden]}{\end{tcolorbox}}\fi

\renewcommand*\contentsname{Table of contents}
{
\hypersetup{linkcolor=}
\setcounter{tocdepth}{3}
\tableofcontents
}
\hypertarget{claim}{%
\subsection{Claim}\label{claim}}

For testing, I think we should replace \(I\) with
\(\tilde{\gamma} = K^{-1} k\), i.e the expansion of \(k\) in the basis
\(K\).

\hypertarget{justification}{%
\subsection{Justification}\label{justification}}

The justification follows the same lines as Kernel DL.

The starting problem we want to solve is:

\begin{equation}\protect\hypertarget{eq-1}{}{\arg\min_{\tilde{x}} \|\varphi(\tilde{y}) - \varphi(D) \tilde{x}\|_2}\label{eq-1}\end{equation}

where \(\varphi\) is the feature map and \(\tilde{y}\) test point (let's
assume it's a single vector).

Kernel DL restricts \(\varphi(D)\) to be in the span of \(\varphi(Y)\),
i.e.~\(\varphi(D) = \varphi(Y) A\) for some A (that is learned), so we
have:

\[\arg\min_{\tilde{x}} \|\varphi(\tilde{y}) - \varphi(Y) A \tilde{x}\|_2\]

Part of \(\varphi(\tilde{y})\) lies is in the span of \(\varphi(Y)\) and
another part is orthogonal to it, and it is clear that only the
component in the span of \(\varphi(Y)\) can ever be minimized with atoms
\(\varphi(Y)\), so without loss of generality we can state the problem
(\ref{eq-1}) as:

\begin{equation}\protect\hypertarget{eq-2}{}{\arg\min_{\tilde{x}} \|\varphi(Y) \tilde{\gamma} - \varphi(Y) A \tilde{x}\|_2}\label{eq-2}\end{equation}

where \(\varphi(Y)\tilde{\gamma}\) is the orthogonal projection of
\(\varphi(\tilde{y})\) onto the span of \(\varphi(Y)\):

\begin{equation}\protect\hypertarget{eq-3}{}{\tilde{\gamma} = \arg\min_{\gamma} \|\varphi(\tilde{y}) - \varphi(Y) \gamma\|_2}\label{eq-3}\end{equation}

With this change, the problem (\ref{eq-2}) becomes similar to the Kernel
DL problem:

\begin{equation}\protect\hypertarget{eq-4}{}{\arg\min_{\tilde{x}} \|\varphi(Y) \left( \tilde{\gamma} - A \tilde{x} \right) \|_2}\label{eq-4}\end{equation}

and the algorithm tries to minimize the right-hand paranthesis
\((\tilde{\gamma} - A \tilde{x})\). Thus, \(\tilde{\gamma}\) is
replacing \(I\) from the training.

Now, coming back at \(\tilde{\gamma}\), this is the least-squares
solution to the problem (\ref{eq-3}):
\begin{equation}\protect\hypertarget{eq-5}{}{\begin{align}
\tilde{\gamma} &= \arg\min_{\gamma} \|\varphi(\tilde{y}) - \varphi(Y) \gamma\|_2 \\
&\overset{(a)}{=} \left( \varphi(Y)^T \varphi(Y) \right)^{-1} \varphi(Y)^T \varphi(\tilde{y}) \\
&= K^{-1} k
\end{align}}\label{eq-5}\end{equation} where \((a)\) is because
\(\varphi(Y)\) is a tall matrix.

Observe that \(\tilde{\gamma}= K^{-1} k\) is just the basis expansion of
\(k\) in \(K\), \[k = K \tilde{\gamma}\]

QED.



\end{document}
