% Options for packages loaded elsewhere
\PassOptionsToPackage{unicode}{hyperref}
\PassOptionsToPackage{hyphens}{url}
\PassOptionsToPackage{dvipsnames,svgnames,x11names}{xcolor}
%
\documentclass[
  letterpaper,
  DIV=11,
  numbers=noendperiod]{scrartcl}

\usepackage{amsmath,amssymb}
\usepackage{lmodern}
\usepackage{iftex}
\ifPDFTeX
  \usepackage[T1]{fontenc}
  \usepackage[utf8]{inputenc}
  \usepackage{textcomp} % provide euro and other symbols
\else % if luatex or xetex
  \usepackage{unicode-math}
  \defaultfontfeatures{Scale=MatchLowercase}
  \defaultfontfeatures[\rmfamily]{Ligatures=TeX,Scale=1}
\fi
% Use upquote if available, for straight quotes in verbatim environments
\IfFileExists{upquote.sty}{\usepackage{upquote}}{}
\IfFileExists{microtype.sty}{% use microtype if available
  \usepackage[]{microtype}
  \UseMicrotypeSet[protrusion]{basicmath} % disable protrusion for tt fonts
}{}
\makeatletter
\@ifundefined{KOMAClassName}{% if non-KOMA class
  \IfFileExists{parskip.sty}{%
    \usepackage{parskip}
  }{% else
    \setlength{\parindent}{0pt}
    \setlength{\parskip}{6pt plus 2pt minus 1pt}}
}{% if KOMA class
  \KOMAoptions{parskip=half}}
\makeatother
\usepackage{xcolor}
\setlength{\emergencystretch}{3em} % prevent overfull lines
\setcounter{secnumdepth}{-\maxdimen} % remove section numbering
% Make \paragraph and \subparagraph free-standing
\ifx\paragraph\undefined\else
  \let\oldparagraph\paragraph
  \renewcommand{\paragraph}[1]{\oldparagraph{#1}\mbox{}}
\fi
\ifx\subparagraph\undefined\else
  \let\oldsubparagraph\subparagraph
  \renewcommand{\subparagraph}[1]{\oldsubparagraph{#1}\mbox{}}
\fi


\providecommand{\tightlist}{%
  \setlength{\itemsep}{0pt}\setlength{\parskip}{0pt}}\usepackage{longtable,booktabs,array}
\usepackage{calc} % for calculating minipage widths
% Correct order of tables after \paragraph or \subparagraph
\usepackage{etoolbox}
\makeatletter
\patchcmd\longtable{\par}{\if@noskipsec\mbox{}\fi\par}{}{}
\makeatother
% Allow footnotes in longtable head/foot
\IfFileExists{footnotehyper.sty}{\usepackage{footnotehyper}}{\usepackage{footnote}}
\makesavenoteenv{longtable}
\usepackage{graphicx}
\makeatletter
\def\maxwidth{\ifdim\Gin@nat@width>\linewidth\linewidth\else\Gin@nat@width\fi}
\def\maxheight{\ifdim\Gin@nat@height>\textheight\textheight\else\Gin@nat@height\fi}
\makeatother
% Scale images if necessary, so that they will not overflow the page
% margins by default, and it is still possible to overwrite the defaults
% using explicit options in \includegraphics[width, height, ...]{}
\setkeys{Gin}{width=\maxwidth,height=\maxheight,keepaspectratio}
% Set default figure placement to htbp
\makeatletter
\def\fps@figure{htbp}
\makeatother

\KOMAoption{captions}{tableheading}
\makeatletter
\makeatother
\makeatletter
\makeatother
\makeatletter
\@ifpackageloaded{caption}{}{\usepackage{caption}}
\AtBeginDocument{%
\ifdefined\contentsname
  \renewcommand*\contentsname{Table of contents}
\else
  \newcommand\contentsname{Table of contents}
\fi
\ifdefined\listfigurename
  \renewcommand*\listfigurename{List of Figures}
\else
  \newcommand\listfigurename{List of Figures}
\fi
\ifdefined\listtablename
  \renewcommand*\listtablename{List of Tables}
\else
  \newcommand\listtablename{List of Tables}
\fi
\ifdefined\figurename
  \renewcommand*\figurename{Figure}
\else
  \newcommand\figurename{Figure}
\fi
\ifdefined\tablename
  \renewcommand*\tablename{Table}
\else
  \newcommand\tablename{Table}
\fi
}
\@ifpackageloaded{float}{}{\usepackage{float}}
\floatstyle{ruled}
\@ifundefined{c@chapter}{\newfloat{codelisting}{h}{lop}}{\newfloat{codelisting}{h}{lop}[chapter]}
\floatname{codelisting}{Listing}
\newcommand*\listoflistings{\listof{codelisting}{List of Listings}}
\makeatother
\makeatletter
\@ifpackageloaded{caption}{}{\usepackage{caption}}
\@ifpackageloaded{subcaption}{}{\usepackage{subcaption}}
\makeatother
\makeatletter
\@ifpackageloaded{tcolorbox}{}{\usepackage[many]{tcolorbox}}
\makeatother
\makeatletter
\@ifundefined{shadecolor}{\definecolor{shadecolor}{rgb}{.97, .97, .97}}
\makeatother
\makeatletter
\makeatother
\ifLuaTeX
  \usepackage{selnolig}  % disable illegal ligatures
\fi
\IfFileExists{bookmark.sty}{\usepackage{bookmark}}{\usepackage{hyperref}}
\IfFileExists{xurl.sty}{\usepackage{xurl}}{} % add URL line breaks if available
\urlstyle{same} % disable monospaced font for URLs
\hypersetup{
  pdftitle={Unsupervised Anomaly Detection with Variational Autoencoders},
  colorlinks=true,
  linkcolor={blue},
  filecolor={Maroon},
  citecolor={Blue},
  urlcolor={Blue},
  pdfcreator={LaTeX via pandoc}}

\title{Unsupervised Anomaly Detection with Variational Autoencoders}
\author{}
\date{1/16/23}

\begin{document}
\maketitle
\ifdefined\Shaded\renewenvironment{Shaded}{\begin{tcolorbox}[boxrule=0pt, borderline west={3pt}{0pt}{shadecolor}, sharp corners, enhanced, frame hidden, breakable, interior hidden]}{\end{tcolorbox}}\fi

\renewcommand*\contentsname{Table of contents}
{
\hypersetup{linkcolor=}
\setcounter{tocdepth}{3}
\tableofcontents
}
\hypertarget{review-papers}{%
\subsection{Review papers}\label{review-papers}}

\hypertarget{an2015}{%
\subsubsection{An2015}\label{an2015}}

\textbf{Variational Autoencoder based Anomaly Detection using
Reconstruction Probability}

\begin{quote}
InProceedings (An2015)\\
An, J. \& Cho, S.\\
Variational Autoencoder based Anomaly Detection using Reconstruction
Probability\\
2015
\end{quote}

\includegraphics{img/2023-01-16-17-17-52.png}

The algorithm:

\begin{itemize}
\item
  VAEs learn a distribution of the inputs
\item
  The latent distribution \(p(z)\) acts as a prior (in Bayesian terms),
  and is the multivariate standard normal and isotropic (i.e.~separable,
  covariance matrix is diagonal)
\item
  \(f(x)\) is the encoder function
\item
  \(g(z)\) is the decoder function
\item
  The decoder function \(g(z)\) maps the distribution of the latent
  variable \(z\) into an output distribution \(p(x|z)\) which should
  resemble the original distribution of \(x\)
\item
  During reconstruction, when we sample a single latent variable \(z\),
  we reconstruct a single vector \(\hat{x}\), so we have a single sample
  of the output distribution \(p(x|z)\)
\item
  Idea for using Reconstruction Probability as an anomaly measure:

  \begin{itemize}
  \item
    sample multiple latent variables \(z^k\), and for each of them
    reconstruct the vector \(\hat{x}^k\)
  \item
    use all the vectors \(\hat{x}^k\) to estimate the probability
    \(p(x|z)\), and then compute the likelihood that the original input
    \(x\) comes from this distribution
  \item
    assuming \(p(x|z)\) is a an isotropic normal distribution, we just
    compute the mean \(\mu = E \lbrace \hat{x} \rbrace\) and covariance
    matrix \(\Sigma\) (diagonal, so basically we compute the variance
    \(\sigma_i^2\) per entry of the vector)
  \item
    the log-likelihood that the original \(x\) is generated by this
    distribution amounts to a weighted \(\ell_2\) norm:

    \[L(x) = \sum_i \frac{(x_i - \mu_i)^2}{\sigma_i^2}\]
  \item
    we use this as an anomaly score: small value = more anomaly, large
    value = more normal
  \item
    small value =\textgreater{} anomaly, because \(x\) does not fit the
    output probability \(p(x|z)\)
  \end{itemize}
\item
  Better than normal AE, because the variances are taken into account
\item
  Perhaps the variances \(\sigma_i^2\) can be used as indicators for
  feature selection?

  \begin{itemize}
  \tightlist
  \item
    or are they just similar to the clones values based on the input
    variances
  \end{itemize}
\end{itemize}

\hypertarget{wievel2019}{%
\subsubsection{Wievel2019}\label{wievel2019}}

\textbf{Continual Learning for Anomaly Detection with Variational
Autoencoder}

\begin{quote}
InProceedings (Wiewel2019)\\
Wiewel, F. \& Yang, B.\\
Continual Learning for Anomaly Detection with Variational Autoencoder\\
ICASSP 2019 - 2019 IEEE International Conference on Acoustics, Speech
and Signal Processing (ICASSP), 2019, 3837-3841
\end{quote}

\begin{itemize}
\item
  Use the full loss function as anomaly score, which includes the
  reconstruction error \textbf{and} the KL distance between the
  distribution of \(z\) and the standard normal prior \(p(z)\)

  \begin{itemize}
  \item
    \begin{quote}
    the so called evidence lower bound (ELBO):
    \includegraphics{img/2023-01-16-17-50-43.png}
    \end{quote}
  \item
    \begin{quote}
    While {[}4, 5, 6, 7{]} use the reconstruction probability E q φ
    (z\textbar x i ) {[}ln p θ (x i \textbar z){]} as the anomaly score,
    we use the ELBO as the anomaly score because it gives slightly
    better results in our experiments.
    \end{quote}
  \end{itemize}
\item
  The ``reconstruction probability'' used in An2015 is just the first
  part of the loss function (ELBO), why not use the full loss, since
  this is what the model was trained to minimize
\end{itemize}



\end{document}
