% Options for packages loaded elsewhere
\PassOptionsToPackage{unicode}{hyperref}
\PassOptionsToPackage{hyphens}{url}
\PassOptionsToPackage{dvipsnames,svgnames,x11names}{xcolor}
%
\documentclass[
  letterpaper,
  DIV=11,
  numbers=noendperiod]{scrartcl}

\usepackage{amsmath,amssymb}
\usepackage{lmodern}
\usepackage{iftex}
\ifPDFTeX
  \usepackage[T1]{fontenc}
  \usepackage[utf8]{inputenc}
  \usepackage{textcomp} % provide euro and other symbols
\else % if luatex or xetex
  \usepackage{unicode-math}
  \defaultfontfeatures{Scale=MatchLowercase}
  \defaultfontfeatures[\rmfamily]{Ligatures=TeX,Scale=1}
\fi
% Use upquote if available, for straight quotes in verbatim environments
\IfFileExists{upquote.sty}{\usepackage{upquote}}{}
\IfFileExists{microtype.sty}{% use microtype if available
  \usepackage[]{microtype}
  \UseMicrotypeSet[protrusion]{basicmath} % disable protrusion for tt fonts
}{}
\makeatletter
\@ifundefined{KOMAClassName}{% if non-KOMA class
  \IfFileExists{parskip.sty}{%
    \usepackage{parskip}
  }{% else
    \setlength{\parindent}{0pt}
    \setlength{\parskip}{6pt plus 2pt minus 1pt}}
}{% if KOMA class
  \KOMAoptions{parskip=half}}
\makeatother
\usepackage{xcolor}
\usepackage[normalem]{ulem}
\setlength{\emergencystretch}{3em} % prevent overfull lines
\setcounter{secnumdepth}{-\maxdimen} % remove section numbering
% Make \paragraph and \subparagraph free-standing
\ifx\paragraph\undefined\else
  \let\oldparagraph\paragraph
  \renewcommand{\paragraph}[1]{\oldparagraph{#1}\mbox{}}
\fi
\ifx\subparagraph\undefined\else
  \let\oldsubparagraph\subparagraph
  \renewcommand{\subparagraph}[1]{\oldsubparagraph{#1}\mbox{}}
\fi


\providecommand{\tightlist}{%
  \setlength{\itemsep}{0pt}\setlength{\parskip}{0pt}}\usepackage{longtable,booktabs,array}
\usepackage{calc} % for calculating minipage widths
% Correct order of tables after \paragraph or \subparagraph
\usepackage{etoolbox}
\makeatletter
\patchcmd\longtable{\par}{\if@noskipsec\mbox{}\fi\par}{}{}
\makeatother
% Allow footnotes in longtable head/foot
\IfFileExists{footnotehyper.sty}{\usepackage{footnotehyper}}{\usepackage{footnote}}
\makesavenoteenv{longtable}
\usepackage{graphicx}
\makeatletter
\def\maxwidth{\ifdim\Gin@nat@width>\linewidth\linewidth\else\Gin@nat@width\fi}
\def\maxheight{\ifdim\Gin@nat@height>\textheight\textheight\else\Gin@nat@height\fi}
\makeatother
% Scale images if necessary, so that they will not overflow the page
% margins by default, and it is still possible to overwrite the defaults
% using explicit options in \includegraphics[width, height, ...]{}
\setkeys{Gin}{width=\maxwidth,height=\maxheight,keepaspectratio}
% Set default figure placement to htbp
\makeatletter
\def\fps@figure{htbp}
\makeatother
\newlength{\cslhangindent}
\setlength{\cslhangindent}{1.5em}
\newlength{\csllabelwidth}
\setlength{\csllabelwidth}{3em}
\newlength{\cslentryspacingunit} % times entry-spacing
\setlength{\cslentryspacingunit}{\parskip}
\newenvironment{CSLReferences}[2] % #1 hanging-ident, #2 entry spacing
 {% don't indent paragraphs
  \setlength{\parindent}{0pt}
  % turn on hanging indent if param 1 is 1
  \ifodd #1
  \let\oldpar\par
  \def\par{\hangindent=\cslhangindent\oldpar}
  \fi
  % set entry spacing
  \setlength{\parskip}{#2\cslentryspacingunit}
 }%
 {}
\usepackage{calc}
\newcommand{\CSLBlock}[1]{#1\hfill\break}
\newcommand{\CSLLeftMargin}[1]{\parbox[t]{\csllabelwidth}{#1}}
\newcommand{\CSLRightInline}[1]{\parbox[t]{\linewidth - \csllabelwidth}{#1}\break}
\newcommand{\CSLIndent}[1]{\hspace{\cslhangindent}#1}

\KOMAoption{captions}{tableheading}
\makeatletter
\makeatother
\makeatletter
\makeatother
\makeatletter
\@ifpackageloaded{caption}{}{\usepackage{caption}}
\AtBeginDocument{%
\ifdefined\contentsname
  \renewcommand*\contentsname{Table of contents}
\else
  \newcommand\contentsname{Table of contents}
\fi
\ifdefined\listfigurename
  \renewcommand*\listfigurename{List of Figures}
\else
  \newcommand\listfigurename{List of Figures}
\fi
\ifdefined\listtablename
  \renewcommand*\listtablename{List of Tables}
\else
  \newcommand\listtablename{List of Tables}
\fi
\ifdefined\figurename
  \renewcommand*\figurename{Figure}
\else
  \newcommand\figurename{Figure}
\fi
\ifdefined\tablename
  \renewcommand*\tablename{Table}
\else
  \newcommand\tablename{Table}
\fi
}
\@ifpackageloaded{float}{}{\usepackage{float}}
\floatstyle{ruled}
\@ifundefined{c@chapter}{\newfloat{codelisting}{h}{lop}}{\newfloat{codelisting}{h}{lop}[chapter]}
\floatname{codelisting}{Listing}
\newcommand*\listoflistings{\listof{codelisting}{List of Listings}}
\makeatother
\makeatletter
\@ifpackageloaded{caption}{}{\usepackage{caption}}
\@ifpackageloaded{subcaption}{}{\usepackage{subcaption}}
\makeatother
\makeatletter
\@ifpackageloaded{tcolorbox}{}{\usepackage[many]{tcolorbox}}
\makeatother
\makeatletter
\@ifundefined{shadecolor}{\definecolor{shadecolor}{rgb}{.97, .97, .97}}
\makeatother
\makeatletter
\makeatother
\ifLuaTeX
  \usepackage{selnolig}  % disable illegal ligatures
\fi
\IfFileExists{bookmark.sty}{\usepackage{bookmark}}{\usepackage{hyperref}}
\IfFileExists{xurl.sty}{\usepackage{xurl}}{} % add URL line breaks if available
\urlstyle{same} % disable monospaced font for URLs
\hypersetup{
  pdftitle={ML/AI in CNC and machining},
  colorlinks=true,
  linkcolor={blue},
  filecolor={Maroon},
  citecolor={Blue},
  urlcolor={Blue},
  pdfcreator={LaTeX via pandoc}}

\title{ML/AI in CNC and machining}
\author{}
\date{2023-01-07}

\begin{document}
\maketitle
\ifdefined\Shaded\renewenvironment{Shaded}{\begin{tcolorbox}[sharp corners, boxrule=0pt, enhanced, breakable, interior hidden, frame hidden, borderline west={3pt}{0pt}{shadecolor}]}{\end{tcolorbox}}\fi

\renewcommand*\contentsname{Table of contents}
{
\hypersetup{linkcolor=}
\setcounter{tocdepth}{3}
\tableofcontents
}
A review of papers on ML / AI approaches in machining and CNC

\hypertarget{review-papers}{%
\subsection{Review papers}\label{review-papers}}

\hypertarget{star-smart-machining-process-using-machine-learning-a-review-and-perspective-on-machining-industry-2018-kim2018_smartmachinereview}{%
\subsubsection{\texorpdfstring{(\(3\star\)) Smart Machining Process
Using Machine Learning: A Review and Perspective on Machining Industry
(2018) (Kim et al.
2018)}{(3\textbackslash star) Smart Machining Process Using Machine Learning: A Review and Perspective on Machining Industry (2018) (Kim et al. 2018)}}\label{star-smart-machining-process-using-machine-learning-a-review-and-perspective-on-machining-industry-2018-kim2018_smartmachinereview}}

Contains a listing of many machining problems where machine learning
algorithms have been used in machining.

\begin{itemize}
\tightlist
\item
  Machine processes: General, milling, drilling
\item
  Purpose: Tool wear and breakage, predict energy consumption, surface
  roughness prediction, process parameter optimization
\item
  Algorithms: SVM and SVR, various NNs, Random forests, linear
  regression, k-NN, depending on task
\end{itemize}

\hypertarget{star-data-driven-cutting-tool-fault-diagnosis-system-using-machine-learning-approach-a-review-tambake2021}{%
\subsubsection{\texorpdfstring{(\(2\star\)) Data Driven Cutting Tool
Fault Diagnosis System Using Machine Learning Approach: A Review
(Tambake, Deshmukh, and Patange
2021)}{(2\textbackslash star) Data Driven Cutting Tool Fault Diagnosis System Using Machine Learning Approach: A Review (Tambake, Deshmukh, and Patange 2021)}}\label{star-data-driven-cutting-tool-fault-diagnosis-system-using-machine-learning-approach-a-review-tambake2021}}

\begin{itemize}
\tightlist
\item
  Poorly written, but contains a list of papers on fault detection
\end{itemize}

\includegraphics{img/2023-01-07-12-19-30.png}

\includegraphics{img/2023-01-07-12-20-27.png}

\hypertarget{star-machine-learning-for-industrial-applications-a-comprehensive-literature-review-bertolini2021}{%
\subsubsection{\texorpdfstring{(\(5\star\)) Machine Learning for
industrial applications: A comprehensive literature review (Bertolini et
al.
2021)}{(5\textbackslash star) Machine Learning for industrial applications: A comprehensive literature review (Bertolini et al. 2021)}}\label{star-machine-learning-for-industrial-applications-a-comprehensive-literature-review-bertolini2021}}

\begin{itemize}
\item
  Good review paper, classifies papers by Application Domain and by ML
  algorithm.
\item
  Not only for CNC / machining, but many types of industrial
  applications
\item
  Large number of papare surveyed
\item
  To come back later
\item
  Hot and not-so-hot topics, clustered:

  \begin{quote}
  Using these metrics, five main clusters can be identified. These are:

  \begin{enumerate}
  \def\labelenumi{\arabic{enumi}.}
  \tightlist
  \item
    \textbf{Question Marks (Low Age and Negative Trend)} -- Recently
    introduced topics, that have not got a follow-up, yet. Thermography
    (THER), Cyber-Physical Systems (CPS), and Design For (D4) belong to
    this category.
  \item
    \textbf{Hot Topics (Low Age and \sout{Negative} Positive Trend)} --
    Very recent topics of booming interest. At present, none of the
    keywords properly belong to this category. Yet, Additive
    Manufacturing (ADD\_MN), Prediction \& Prognostic (PR\_PR), and
    Industry 4.0 (I4.0) are those who come closest to this category. For
    this reason, they have been labeled as `new promises'.
  \item
    \textbf{Consolidated (Medium Age and Stable Trend)} -- Not recent
    topics, which are still studied, but without the initial spike of
    interests. Topics such as Supply Chain Management (SCMI), Flexible
    Manufacturing Systems (FMS), Inventory Control (INV\_CTRI), and Tool
    Monitoring (TLL\_MN) belong to this category.
  \item
    \textbf{Stars (High Age and Positive Trend)} -- Old and consolidated
    topics that are still attracting increasing research interest.
    Topics such as Diagnosis and Fault Detection (DG\_FLT),
    Manufacturing Process (MN\_PR), Intelligent Manufacturing (INT\_MN),
    and Big Data analysis (BD\_DM) certainly belong to this class.
    Probably, Simulation (SIM) and the Internet of Things (IoT) are on
    their way to become stars.
  \item
    \textbf{Obsoletes (High Age and Negative Trend)} -- Old topics that
    have never received much scientific interest and that have almost
    disappeared from the technical literature. Due to the recent
    introduction of ML, for operation management, no keywords can be
    classified as obso­ letes yet. However, Order Management (OM) and,
    probably, also Feature Extraction (FT\_EX) are moving toward this
    class.
  \end{enumerate}
  \end{quote}

  \includegraphics{img/2023-01-07-12-44-41.png}
\end{itemize}

\includegraphics{img/2023-01-07-12-32-41.png}

\includegraphics{img/2023-01-07-12-38-32.png} \ldots{}

\begin{itemize}
\item
  Anomaly detection examples:

  \includegraphics{img/2023-01-07-12-48-20.png}

  \includegraphics{img/2023-01-07-12-49-39.png}
\item
  Sample variables commonly used in datasets:

  \begin{quote}
  Table 4, which provides some indications concerning the variables that
  are commonly used per each application domain and sub-area
  \end{quote}

  \includegraphics{img/2023-01-07-12-53-08.png}
\end{itemize}

\hypertarget{technical-papers}{%
\subsection{Technical papers}\label{technical-papers}}

\hypertarget{wear-characterization-of-the-cutting-tool-in-milling-processes-using-shape-and-texture-descriptors-phd-thesis-2017ordas2017_wearcharactphd}{%
\subsubsection{Wear Characterization of the Cutting Tool in Milling
Processes using Shape and Texture Descriptors (PhD thesis,
2017)(García-Ordás
2017)}\label{wear-characterization-of-the-cutting-tool-in-milling-processes-using-shape-and-texture-descriptors-phd-thesis-2017ordas2017_wearcharactphd}}

\begin{itemize}
\item
  PhD thesis which proposes and evaluates some image-based descriptors
  to characterize tool wear, using a cheap Raspberry Pi + camera setup
  which captures images of the cutting tool.
\item
  Investigative / no remarkable results.
\end{itemize}

\hypertarget{surface-roughness-diagnosis-in-hard-turning-using-acoustic-signals-and-support-vector-machine-a-pca-based-approach-papandrea2020}{%
\subsubsection{Surface roughness diagnosis in hard turning using
acoustic signals and support vector machine: A PCA-based approach
(Papandrea et al.
2020)}\label{surface-roughness-diagnosis-in-hard-turning-using-acoustic-signals-and-support-vector-machine-a-pca-based-approach-papandrea2020}}

\begin{itemize}
\tightlist
\item
  Surface roughness classification, based on acoustic signals during
  cutting.
\item
  Use STFT followed by PCA per coefficients, and SVM for classification.
\item
  Tested on CNC, with stock microphone
\item
  Some complicated experimental machining setups, several parameters in
  the process (rotating speed, feed rate). The experimental setups
  depend on many factors.
\item
  Results weak. Some PCA coeffs are correlated with roughness, and can
  be clustered consistently into 3 groups, which can then be identified
  in test sets.
\item
  More investigative / basic research, no remarkable results.
\end{itemize}

\includegraphics{img/2023-01-07-12-16-33.png}

\hypertarget{refs}{}
\begin{CSLReferences}{1}{0}
\leavevmode\vadjust pre{\hypertarget{ref-Bertolini2021}{}}%
Bertolini, Massimo, Davide Mezzogori, Mattia Neroni, and Francesco
Zammori. 2021. {``Machine {Learning} for Industrial Applications: {A}
Comprehensive Literature Review.''} \emph{Expert Systems with
Applications} 175 (August): 114820.
\url{https://doi.org/10.1016/j.eswa.2021.114820}.

\leavevmode\vadjust pre{\hypertarget{ref-Ordas2017_WearCharactPhD}{}}%
García-Ordás, Teresa María. 2017. {``Wear Characterization of the
Cutting Tool in Milling Processes Using Shape and Texture
Descriptors.''} PhD thesis. \url{https://doi.org/10.18002/10612/6915}.

\leavevmode\vadjust pre{\hypertarget{ref-Kim2018_SmartMachineReview}{}}%
Kim, Dong-Hyeon, Thomas J. Y. Kim, Xinlin Wang, Mincheol Kim, Ying-Jun
Quan, Jin Woo Oh, Soo-Hong Min, et al. 2018. {``Achine Learning in Cnc
Machining: Best Practices.''} \emph{International Journal of Precision
Engineering and Manufacturing-Green Technology} 5 (4): 555--68.
\url{https://doi.org/10.1007/s40684-018-0057-y}.

\leavevmode\vadjust pre{\hypertarget{ref-Papandrea2020}{}}%
Papandrea, Pedro J., Edielson P. Frigieri, Paulo Roberto Maia, Lucas G.
Oliveira, and Anderson P. Paiva. 2020. {``Surface Roughness Diagnosis in
Hard Turning Using Acoustic Signals and Support Vector Machine: {A}
{PCA}-Based Approach.''} \emph{Applied Acoustics} 159 (February):
107102. \url{https://doi.org/10.1016/j.apacoust.2019.107102}.

\leavevmode\vadjust pre{\hypertarget{ref-Tambake2021}{}}%
Tambake, Nagesh R., Bhagyesh B. Deshmukh, and Abhishek D. Patange. 2021.
{``Data {Driven} {Cutting} {Tool} {Fault} {Diagnosis} {System} {Using}
{Machine} {Learning} {Approach}: {A} {Review}.''} \emph{Journal of
Physics: Conference Series} 1969 (1): 012049.
\url{https://doi.org/10.1088/1742-6596/1969/1/012049}.

\end{CSLReferences}



\end{document}
